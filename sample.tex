
%%%%%%%%%%%%%%%%%
% This is a sample CV template created using altacv.cls
% (v1.6, 21 May 2021) written by LianTze Lim (liantze@gmail.com). Now compiles with pdfLaTeX, XeLaTeX and LuaLaTeX.
%
%% It may be distributed and/or modified under the
%% conditions of the LaTeX Project Public License, either version 1.3
%% of this license or (at your option) any later version.
%% The latest version of this license is in
%%    http://www.latex-project.org/lppl.txt
%% and version 1.3 or later is part of all distributions of LaTeX
%% version 2003/12/01 or later.
%%%%%%%%%%%%%%%%

%% Use the "normal photo" option if you want a normal photo instead of cropped to a circle
% \documentclass[10pt,a4paper,normalphoto]{altacv}

\documentclass[10pt,a4paper,ragged2e,withhyper]{altacv}
%% AltaCV uses the fontawesome5 and packages.
%% See http://texdoc.net/pkg/fontawesome5 for a full list of symbols.

% Change the page layout if you need to
\geometry{left=1.25cm,right=1.25cm,top=1.5cm,bottom=1.5cm,columnsep=1.2cm}

% The paracol package lets you typeset columns of text in parallel
\usepackage{paracol}
\usepackage[utf8]{inputenc}

% Change the font if you want to, depending on whether
% you're using pdflatex or xelatex/lualatex
\ifxetexorluatex
  % If using xelatex or lualatex:
  \setmainfont{Roboto Slab}
  \setsansfont{Lato}
  \renewcommand{\familydefault}{\sfdefault}
\else
  % If using pdflatex:
  \usepackage[rm]{roboto}
  \usepackage[defaultsans]{lato}
  % \usepackage{sourcesanspro}
  \renewcommand{\familydefault}{\sfdefault}
\fi

% Change the colours if you want to
\definecolor{SlateGrey}{HTML}{2E2E2E}
\definecolor{LightGrey}{HTML}{666666}
\definecolor{DarkPastelRed}{HTML}{450808}
\definecolor{PastelRed}{HTML}{8F0D0D}
\definecolor{GoldenEarth}{HTML}{E7D192}
\colorlet{name}{black}
\colorlet{tagline}{PastelRed}
\colorlet{heading}{DarkPastelRed}
\colorlet{headingrule}{GoldenEarth}
\colorlet{subheading}{PastelRed}
\colorlet{accent}{PastelRed}
\colorlet{emphasis}{SlateGrey}
\colorlet{body}{LightGrey}

% Change some fonts, if necessary
\renewcommand{\namefont}{\Huge\rmfamily\bfseries}
\renewcommand{\personalinfofont}{\footnotesize}
\renewcommand{\cvsectionfont}{\LARGE\rmfamily\bfseries}
\renewcommand{\cvsubsectionfont}{\large\bfseries}


% Change the bullets for itemize and rating marker
% for \cvskill if you want to
\renewcommand{\itemmarker}{{\small\textbullet}}
\renewcommand{\ratingmarker}{\faCircle}

%% Use (and optionally edit if necessary) this .cfg if you
%% want to use an author-year reference style like APA(6)
%% for your publication list
% When using APA6 if you need more author names to be listed
% because you're e.g. the 12th author, add apamaxprtauth=12
\usepackage[backend=biber,style=apa6,sorting=ydnt]{biblatex}
\defbibheading{pubtype}{\cvsubsection{#1}}
\renewcommand{\bibsetup}{\vspace*{-\baselineskip}}
\AtEveryBibitem{\makebox[\bibhang][l]{\itemmarker}}
\setlength{\bibitemsep}{0.25\baselineskip}
\setlength{\bibhang}{1.25em}


%% Use (and optionally edit if necessary) this .cfg if you
%% want an originally numerical reference style like IEEE
%% for your publication list
% \usepackage[backend=biber,style=ieee,sorting=ydnt]{biblatex}
%% For removing numbering entirely when using a numeric style
\setlength{\bibhang}{1.25em}
\DeclareFieldFormat{labelnumberwidth}{\makebox[\bibhang][l]{\itemmarker}}
\setlength{\biblabelsep}{0pt}
\defbibheading{pubtype}{\cvsubsection{#1}}
\renewcommand{\bibsetup}{\vspace*{-\baselineskip}}


%% sample.bib contains your publications
\addbibresource{sample.bib}

\newcommand*\leftright[3]{%
	\leavevmode
	\rlap{#2}%
	\hspace{#1}%
	#3}

\begin{document}

\name{Beno{\^\i}t Verhaeghe}
\tagline{Ph.D. in Software Engineering}
%% You can add multiple photos on the left or right
\photoR{2.8cm}{profil}

\personalinfo{%
  % Not all of these are required!
  \email{work@badetitou.fr}
  \phone{+33 6 58 33 53 74}
  \mailaddress{34b, Avenue Jean Jaurès}
  \location{38600 Fontaine, FRANCE}
  \homepage{badetitou.fr}
  \twitter{@badetitou}
  \linkedin{benoitverhaeghe}
  \github{badetitou}
  \orcid{0000-0002-4588-2698}
}

\makecvheader
%% Depending on your tastes, you may want to make fonts of itemize environments slightly smaller
% \AtBeginEnvironment{itemize}{\small}

%% Set the left/right column width ratio to 6:4.
\columnratio{0.6}

% Start a 2-column paracol. Both the left and right columns will automatically
% break across pages if things get too long.
\begin{paracol}{2}
\cvsection{Experience}

\cvevent{Scientific Project Manager}{Berger-Levrault}{January 2024}{Lyon, France}

\begin{itemize}
  \item Manage research projects about 
  \item GreenIT 
  \item AI for code
  \item Code migration and modernization 
  \item Tests Generation
  \end{itemize}
  
  Also Managing a team of Researchers, Engineers, and PhD students
  
  Collaborating with several Research labs And with Industrial Business Units

\divider

\cvevent{R\&D Software Engineer}{Berger-Levrault}{November 2021 -- December 2023}{Lyon, France}

\begin{itemize}
  \item Migrating applications to Angular
  \item Manage Ph.D. Students and interns
  \item Writing research papers
\end{itemize}

\divider

\cvevent{Ph.D. Software Engineer}{Berger-Levrault/Inria RMoD}{January 2019 -- October 2021}{Montpellier/Lille, France}

\begin{itemize}
  \item Migrating GWT applications to Angular
  \item Set up hybrid architecture (mixing GWT and Angular)
  \item Writing research papers
  \item Teaching
\end{itemize}

\divider

\cvevent{R\&D Software Engineer}{Berger-Levrault}{March 2018 -- December 2018}{Montpellier, France}

\begin{itemize}
  \item Designing tools to migrate GWT applications to Angular
  \item Analyse the company's source code
\end{itemize}

\divider

\cvevent{Research Internship}{Inria RMoD}{May -- September 2017}{Villeneuve d'Ascq, France}


\cvsection{Projects}

\cvevent{VSCode plugin --- Pharo Language Support}{Personnal project}{December 2020 -- Today}{}

\begin{itemize}
  \item Write Pharo code in VSCode
  \item Debug Pharo code in VSCode
  \item Use VSCode built-in notebook with Pharo
  \item +1500 installs in VSCode marketplace
\end{itemize}

\divider


\cvevent{Casino}{Inria RMoD/Berger-Levrault}{March 2018 --- December 2023}{Lille, France}

\begin{itemize}
  \item Automatic migration of application GUI
\end{itemize}

\switchcolumn

\cvsection{Success}

\cvachievement{\faTrophy}{Open Science Thesis Award}{My thesis work was awarded the \href{https://www.enseignementsup-recherche.gouv.fr/fr/remise-des-premiers-prix-science-ouverte-de-la-these-97810}{\color{blue}\underline{Open Science Thesis Award}} by the Ministry of Higher Education and Research}

\divider

\cvachievement{\faTrophy}{Third place Innovation Award}{\href{https://marketplace.visualstudio.com/items?itemName=badetitou.pharo-language-server}{\color{blue}\underline{Pharo Language Server}} was elected third at the Innovation Award of ESUG'2022}

\divider

\cvachievement{\faTrophy}{Best Paper Award}{Migrating GWT to Angular 6 using MDE @ Sattose}

\divider

\cvachievement{\faTrophy}{Second place Innovation Award}{\href{https://badetitou.fr/projects/SmartTest/}{\color{blue}\underline{SmartTest}} was elected second at the Innovation Award of ESUG'2017}

% \divider

% \cvachievement{\faHeartbeat}{Another achievement}{more details about it of course}

\cvsection{Strengths}

\cvtag{Pharo}
\cvtag{MDE}
\cvtag{Java/JEE}
\cvtag{Angular} \\
\cvtag{GWT}
\cvtag{Spring}
\cvtag{Rest}

\divider

\cvtag{Git}
\cvtag{Jenkins}
\cvtag{GitHub actions}

\cvsection{Languages}

\leftright{5em}{\textbf{French}}{Mother tongue} \par
\divider

\leftright{5em}{\textbf{English}}{Fluent}  \par

%% Yeah I didn't spend too much time making all the
%% spacing consistent... sorry. Use \smallskip, \medskip,
%% \bigskip, \vspace etc to make adjustments.
\medskip

\cvsection{Education}

\cvevent{Ph.D. in Software Engineering}{University of Lille / Berger-Levrault}{January 2019 -- October 2021}{}
Incremental Approach for Application GUI Migration using Metamodel

\divider

\cvevent{Engineering Degree in Software Engineering and Statistic}{Polytech Lille}{September 2015 -- July 2018}{}

\divider

\cvevent{University Diploma of Technology in Computer Science}{IUT A - University of Lille}{Sept 2014 -- June 2015}{}

% \divider

\cvsection{Hobbies}

\cvtag{Board game}
\cvtag{Open-Source Contributor}


\end{paracol}

\newpage

\cvsection{Long CV}

\cvsection{Experience}

\cvevent{R\&D Software Engineer}{Berger-Levrault}{November 2021 -- Today}{Lyon, France}

\begin{itemize}
  \item Migrating applications to Angular (Casino Project -- MDE)
  \item Analyse code quality 
  \item Manage Ph.D. Students and interns
  \item Writing research papers
  \item Follow-up academic partnership
\end{itemize}

\divider

\cvevent{Ph.D. Software Engineer}{Berger-Levrault/Inria RMoD}{January 2019 -- October 2021}{Montpellier/Lille, France}

\begin{itemize}
  \item Migrating GWT applications to Angular
  \item Set up hybrid architecture (mixing GWT and Angular together)
  \item Writing research papers
  \item Teaching
  \item Writing blog post about the Moose platform
  \begin{itemize}
    \item \url{https://modularmoose.org/2022/10/27/test-your-moose-code-using-ci.html}
    \item \url{https://modularmoose.org/2021/10/01/migrating-internationalization.html}
    \item \url{https://modularmoose.org/2021/07/19/automatic-metamodel-documentation-generation.html}
    \item \url{https://modularmoose.org/2021/05/15/connecting-meta-models.html}
  \end{itemize}
\end{itemize}

\divider

\cvevent{R\&D Software Engineer}{Berger-Levrault}{March 2018 -- December 2018}{Montpellier, France}

\begin{itemize}
  \item Creation of tools to migrate GWT applications to Angular technology
  \item Meta-modeling
  \item Code analysis
  \item State of the art
\end{itemize}

\divider

\cvevent{Research Internship}{Inria RMoD}{May -- September 2017}{Villeneuve d'Ascq, France}

\begin{itemize}
  \item Learning the Pharo programming language
  \item Understanding tests selection
  \item Developing an automatic tests selection tool
  \item Test the tool with the Pharo community
  \item Integration of the tool in the Pharo core
  \item Writing research papers
\end{itemize}

\divider

\cvevent{Software Developer}{University of Emden,  Departement Computer Science}{Avril -- July 2015}{Emden, Allemagne}

\begin{itemize}
  \item Developing a Java application using the OPC-UA protocol
\end{itemize}

\cvsection{Projects}

\cvevent{VSCode plugin --- Pharo Language Support}{Personnal project}{January 2021}{}

\begin{itemize}
  \item Write Pharo code in VSCode
  \item Debug Pharo code in VSCode
  \item Use VSCode built-in notebook with Pharo
  \item +1500 installs in VSCode marketplace
\end{itemize}

\divider


\cvevent{Casino}{Inria RMoD/Berger-Levrault}{March 2018 --- today}{Montpellier, France}

\begin{itemize}
  \item Automatic migration of application GUI
  \item 6 importers and 5 exporters
\end{itemize}


\divider

\cvevent{Pharo LibVLC}{Personnal project}{}{}

\href{https://github.com/badetitou/Pharo-LibVLC}
{\color{blue}\underline{Pharo LibVLC}} is an FFI binding of the VLC library with Pharo.


\divider

\cvevent{SmartTest}{Inria RMoD}{July 2017 --- July 2018}{Lille, France}
\begin{itemize}
  \item Automatic selection of tests in Pharo
  \item Development of different test execution strategies
\end{itemize}

\cvsection{teaching}

\cvevent{Software architecture}{Polytech Lille}{2019 --- 2021}{France}

\divider

\cvevent{Software evolution and metamodeling}{University of Montpellier}{2019 --- 2021}{Montpellier, France}

\divider

\cvevent{Basic Algorithms}{University of Montpellier}{2019 --- 2021}{Montpellier, France}

\cvsection{Education}

\cvevent{Ph.D. in Software Engineering}{University of Lille / Berger-Levrault}{January 2019 -- October 2021}{}
Incremental Approach for Application GUI Migration using Metamodel


\textbf{Academic supervisors:}\\
Nicolas Anquetil -- Associate professor -- Université de Lille\\
Anne Etien -- Full professor -- Université de Lille\\
\textbf{Industrial supervisor:} Abderrahmane Seriai -- Berger-Levrault


\textbf{Reviewers:}\\
Salah Sadou -- Full professor -- Université Bretagne Sud\\
Jean-Rémy Falleri -- Associate professor -- Université de Bordeaux Sud\\

\textbf{President:}\\
Franck Barbier -- Full professor -- Université de Pau et des Pays de l’Adour


\divider

\cvevent{Engineering Degree in Software Engineering and Statistic}{Polytech Lille}{September 2015 -- July 2018}{}

\textbf{Supervisors:} Anne Etien, Nicolas Anquetil, Laurent Deruelle, and Abderrahmane Seriai

\divider

\cvevent{University Diploma of Technology in Computer Science}{IUT A - University of Lille}{Sept 2014 -- June 2015}{}


\cvsection{Publications}

\nocite{*}

% \printbibliography[heading=pubtype,title={\printinfo{\faBook}{Books}},type=book]

% \divider

\printbibliography[heading=pubtype,title={\printinfo{\faFile*[regular]}{Journal Articles}},type=article]

\divider

\printbibliography[heading=pubtype,title={\printinfo{\faUsers}{Conference Proceedings}},type=inproceedings]

\divider

\printbibliography[heading=pubtype,title={\printinfo{\faUsers}{PhD thesis}},type=thesis]

\end{document}
